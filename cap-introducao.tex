% Organização sugerida para uma introdução de projeto de pesquisa

\chapter{Introdução}

Os criptoativos, também conhecidos como criptomoedas ou moedas digitais, são representações digitais de valor que utilizam criptografia para garantir transações seguras e controlar a criação de novas unidades \cite{Yetmar2023}. Ao contrário das moedas governamentais tradicionais, os criptoativos operam em redes descentralizadas, geralmente baseadas em tecnologia de \textit{Blockchain}, onde a validação das transações é realizada pelos próprios participantes da rede através de um consenso distribuído. Este ambiente transparente, seguro e descentralizado existe graças a arquiteturas de \textit{software} complexas e dedicadas ao propósito de escalabilidade, segurança e autonomia do ativo. Diante da versatilidade dos criptoativos, este projeto de pesquisa visa esclarecer o entendimento popular dos criptoativos enquanto expõe uma análise técnica e econômica do mesmo. 

Na seção \ref*{sec:bitcoin} utilizamos do \textit{Bitcoin} como exemplo, expomos o contexto de aplicação deste ativo, seu posicionamento sobre o dinheiro tradicional e como podemos utilizar do ambiente das criptos para buscar independência monetária. É apresentado também a tecnologia de encadeamento e rede em que o Bitcoin atua — a \textit{Blockchain}, a tecnologia Prova de Trabalho, ou \textit{Proof of work}, utilizada como validador de novos blocos e a técnica utilizada na assinatura de transações dentro da \textit{Blockchain} — a criptografia de chave pública e privada.

Pela seção \ref*{sec:austriaca} são adotadas as definições de dinheiro, economia e capital da Escola Austríaca de Economia. Será utilizado dos conceitos financeiros para definir quão bem o criptoativo consegue servir as atividades econômicas.

Durante a seção \ref*{sec:dinheiro} é feita uma revisão do livro ``\textit{Bitcoin}, A Moeda Na Era Digital'' do mestre brasileiro em economia Fernando Ulrich. Neste livro o autor também revisa o posicionamento da Escola Austríaca de Economia diante do conceito de dinheiro e moeda, traz contexto histórico sobre o desenvolvimento do \textit{Bitcoin} e defende a utilização do criptoativo como moeda de troca legítimo.


Por fim, no capítulo \ref*{sec:educativo}, propomos a ideia do desenvolvimento de uma criptomoeda capaz de conteinerizar e abstrair os conceitos básicos do ensino de economia, provendo autonomia para professores simularem ambientes econômicos. Assim demonstrando e aplicando o conhecimento teórico da ciência financeira.

