%%Altere as informações do resumo%%
\noindent{JUNIOR; HEBER. \textbf{\imprimirtitulo}. Trabalho de Conclusão de Curso (Graduação) - \imprimirinstituicao. \imprimirlocal, \imprimirdata.}
\par
\begin{resumo}
	%%Escreva seu resumo na língua vernácula aqui%%
Nos últimos anos, as criptomoedas têm conquistado seu caminho no cenário financeiro global, trazendo uma nova maneira na realização de transações econômicas. A ascensão das moedas digitais como o\textit{Bitcoin} e \textit{Ethereum} trouxeram consigo não apenas uma revolução na tecnologia financeira, mas também uma onda de especulação, debate e inovação. No entanto, além do seu potencial como veículo de investimento e meio de troca, o mundo dos criptoativos abre portas também para a propósitos educacionais. Este artigo explora a possibilidade e a tecnologia necessária no desenvolvimento de um criptoativo educacional, voltado a refinar a maneira em que as pessoas aprendem os princípios fundamentais da ciência econômica e financeira. Este projeto se utiliza das bases das moedas digitais, suas tecnologias empregadas e seu impacto sobre a economia tradicional, se apoiando também na definição do dinheiro sob a visão dos autores presentes na Escola Austríaca de Economia.
\vspace{\onelineskip} \\
	%%Adicione as palavras chaves após os dois pontos '':''%%
\noindent
\textbf{Palavras-chaves}: Tecnologia financeira; Economia; \textit{Blockchain}; \textit{Bitcoin}; Educação financeira; Criptoativos; \textit{DeFi}.

\end{resumo}
