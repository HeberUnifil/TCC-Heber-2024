\chapter{Conceito de possível aplicação da arquitetura de criptoativos}

Por viés de exemplificar a utilidade da arquitetura de criptoativos e das suas funções como moeda de fato, este projeto pretende pesquisar e, se possível, teorizar o conceito de um protocolo de criptoativos dedicado a auxiliar no ensino de finanças economia, conforme explicado a seguir.

\section*{Um criptoativo educativo}
\label{sec:educativo}

Conforme definido nos capítulos anteriores, a arquitetura de moedas digitais vestem precisamente o papel de dinheiro. Ao passo que desenvolver sistemas descentralizados torna-se cada vez mais flexível, e melhor aplicável a ideia do desenvolvimento de um criptoativo que congregasse regras econômicas de forma didática e controlada.

Este projeto tem o viés de, futuramente, estipular quais seriam as necessidades, objetivos, infraestruturas, linguagens, arquiteturas, e características base deste ativo. Em caso de sucesso da orquestração e desenvolvimento, é interessante a publicação de código aberto deste projeto, promovendo maior ciência e margem para melhorias.








