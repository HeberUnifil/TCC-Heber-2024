\chapter{Metodologia de Pesquisa}
A pesquisa foi conduzida através de uma extensa revisão bibliográfica, diante das revistas acadêmicas ACM, Semantic Scholar, IEEE, em busca de artigos apresentando o estado da arte na área de criptoativos. Foram coletados 86 artigos selecionados inicialmente pelas palavras chaves Criptoativos, \textit{Bitcoin}, \textit{Smart Contracts}, \textit{Blockchain}, \textit{DeFi} e \textit{Web 3}. Utilizamos a revisão destes artigos de modo a buscar o estado da arte documentado diante dos criptoativos.

Apos a coleta inicial dos artigos, o primeiro processo de filtragem se passou pela leitura do resumo dos artigos, mantendo apenas os documentos que retratavam a utilização e o impacto socio-econômico dos criptoativos e a exploração do conceito de finanças descentralizadas. Esta primeira filtragem nos retornou 58 registros. 

O segundo processo de filtragem passou-se pela leitura do conteúdo de cada artigo, buscando apenas os relacionamentos dentre os termos técnicos de economia paralelamente ao funcionamento dos ativos digitais. Diante deste processamento de documentação, foi constatada a necessidade da busca bibliográfica de referências da Escola Austríaca de Economia — devido à semelhança de comportamento agnóstico ao estado tanto desta vertente acadêmica quanto das finanças descentralizadas — no que foi considerada busca de literatura de seus principais autores. Esta segunda filtragem nos retornou 7 livros e 27 artigos.

Por fim, a partir da leitura de toda a documentação coletada, foi registrado o estado da arte na utilização de criptoativos, dentre seus serviços, propostas e ferramentas. Foi registrada também a atuação técnica da moeda digital mais utilizada atualmente, o \textit{Bitcoin}\footnote{Conforme o indexador de criptoativos \href{https://www.coingecko.com/pt}{\textit{CoinGecko}}, acessado 22/05/2024}, registrado o comportamento do ativo diante da visão da Escola Austríaca de Economia e documentado as comparações dos ativos digitais diante do ouro e o papel-moeda.







