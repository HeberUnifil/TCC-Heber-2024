\chapter{Resultados esperados}

Tendo posto o entendimento do funcionamento dos criptoativos, seus objetivos, suas problemáticas e seu devido posicionamento como dinheiro, este trabalho visa clarificar a população geral a possibilidade da adoção do ambiente das finanças descentralizadas (\textit{DeFi}) e apresentar uma centelha de liberdade e autonomia financeira ao leitor enquanto olha para fora do ambiente tradicional de mercado.

Este projeto almeja também a possibilidade de simular e/ou conteinerizar os movimentos da economia dentro de um protocolo \textit{DeFi}, que seria posteriormente dedicado a usar tal simulação para o ensino de finanças.

O funcionamento deste protocolo deverá seguir as movimentações econômicas baseadas nos autores austríacos mencionados previamente enquanto preserva as qualidades de segurança do \textit{Bitcoin} e provê autonomia para professores manipularem o ambiente simulado e experienciar movimentações de resposta teorizadas diante da economia.

\section{Limitações do trabalho}

O projeto de pesquisa passa pela limitação de que, no estado atual de atividade entre meio destes ativos, o lançamento de novas moedas digitais são extremamente frequentes, gerando constante necessidade de reindexação da funcionalidade e atuação de cada nova moeda.

