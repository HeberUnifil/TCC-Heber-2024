
%Para melhor utilização da ferramenta, utilize o pdfLaTeX+MakeIndex+BibTeX%
\documentclass[article,12pt,oneside,a4paper,english,brazil]{unifil}

% IMPORTS
\usepackage{placeins}
\usepackage{pgfgantt}

\DeclareUnicodeCharacter{00A0}{ }

% Caso queira usar um diretório diferente pra imagem, tire o comentário da proxima linha 
% \graphicspath{ {/home/hug/MonografeArq/}}
\titulo{Um Estudo Econômico-Computacional Aplicado Sobre Criptoativos \\ 
\fontsize{12}{14}\selectfont An Applied Economic-Computational Study
On Cryptoassets}
\autor{Heber José da Silva Junior\thanks{Centro Universitário Filadélfia de Londrina - UniFil}
\\Mario Henrique Akihiko da Costa Adaniya\thanks{Centro Universitário Filadélfia de Londrina - UniFil}}
\instituicao{Centro Universitário Filadélfia}
\local{Londrina}
\predate{}
\postdate{}
\date{}

\makeatletter
\let\@fnsymbol\@arabic
\makeatother

\usepackage{caption}
\captionsetup[figure]{slc=off}

\usepackage{titling}
\setlength{\droptitle}{-3cm}
\preauthor{\begin{flushright}
\large \lineskip 0.5em%
}
\postauthor{\end{flushright}}

\usepackage{graphicx}
\usepackage{titlesec}
\titleformat{\section}
{\normalfont\fontsize{14}{15}\bfseries}{\thesection}{1em}{}

\SingleSpacing

\begin{document}
\frenchspacing
\maketitle 
\normalsize

\fontsize{10}{1}\selectfont
\section*{Resumo}
Nos últimos anos, as criptomoedas têm conquistado seu caminho no cenário financeiro global, trazendo uma nova maneira na realização de transações econômicas. A ascensão das moedas digitais como o \textit{Bitcoin} e \textit{Ethereum} trouxeram consigo não apenas uma revolução na tecnologia financeira, mas também uma onda de especulação, debate e inovação. No entanto, além do seu potencial como veículo de investimento e meio de troca, o mundo dos criptoativos abre portas também para a propósitos educacionais. Este artigo explora a possibilidade e a tecnologia necessária no desenvolvimento de um criptoativo educacional, voltado a refinar a maneira em que as pessoas aprendem os princípios fundamentais da ciência econômica e financeira. Este projeto se utiliza das bases das moedas digitais, suas tecnologias empregadas e seu impacto sobre a economia tradicional, se apoiando também na definição do dinheiro sob a visão dos autores presentes na Escola Austríaca de Economia.\\
\vspace{\onelineskip} \\
\noindent
\textbf{Palavras-chave}: Tecnologia financeira; Economia; \textit{Blockchain}; \textit{Bitcoin}; Educação financeira; Criptoativos; \textit{DeFi}.



\section*{Abstract}
\begin{otherlanguage*}{english}
%resumo em ingles%
In recent years, cryptocurrencies have burst onto the global financial scene, bringing with them a new way of conducting economic transactions. The rise of digital currencies such as bitcoin and ethereum has not only revolutionized financial technology, but also sparked a wave of speculation, debate, and innovation. But beyond its potential as an investment vehicle and medium of exchange, the world of cryptoassets also opens doors for educational purposes. This article explores the possibility and the technology required to develop an educational cryptoasset aimed at refining the way people learn the basic principles of economics and finance. This project uses the fundamentals of digital currencies, their technologies and their impact on the traditional economy, and also draws on the definition of money from the point of view of the authors of the Austrian School of Economics.\\
\vspace{\onelineskip}\\
\noindent
\textbf{Keywords}: Financial technology; Economics; \textit{Blockchain}; \textit{Bitcoin}; Financial education; Cryptoassets; \textit{DeFi}
\end{otherlanguage*}
\clearpage

%texto%
\textual
\fontsize{12}{7}\selectfont
\begin{Spacing}{1.5}

% Organização sugerida para uma introdução de projeto de pesquisa

\chapter{Introdução}

Os criptoativos, também conhecidos como criptomoedas ou moedas digitais, são representações digitais de valor que utilizam criptografia para garantir transações seguras e controlar a criação de novas unidades. Ao contrário das moedas governamentais tradicionais, os criptoativos operam em redes descentralizadas, geralmente baseadas em tecnologia de Blockchain, onde a validação das transações é realizada pelos próprios participantes da rede através de um consenso distribuído. Este ambiente transparente, seguro e descentralizado existe graças a arquiteturas de \textit{software} complexas e dedicadas ao propósito de escalabilidade, segurança e autonomia do ativo. Diante da versatilidade dos criptoativos, este projeto de pesquisa visa esclarecer o entendimento popular dos criptoativos enquanto expõe uma análise técnica e econômica do mesmo. 

Na seção \ref*{sec:bitcoin} utilizamos do Bitcoin como exemplo, expomos o contexto de aplicação deste ativo, seu posicionamento sobre o dinheiro tradicional e como podemos utilizar do ambiente das criptos para buscar independência monetária. É apresentado também a tecnologia de encadeamento e rede em que o Bitcoin atua — a Blockchain, a tecnologia Prova de Trabalho, ou \textit{Proof of work}, utilizada como validador de novos blocos e a técnica utilizada na assinatura de transações dentro da Blockchain — a criptografia de chave pública e privada.

Pela seção \ref*{sec:austriaca} são adotadas as definições de dinheiro, economia e capital da Escola Austríaca de Economia. Será utilizado dos conceitos financeiros para definir quão bem o criptoativo consegue servir as atividades econômicas.

Durante a seção \ref*{sec:dinheiro} é feita uma revisão do livro "Bitcoin, A Moeda Na Era Digital" do mestre brasileiro em economia Fernando Ulrich. Neste livro o autor também revisa o posicionamento da Escola Austríaca de Economia diante do conceito de dinheiro e moeda, traz contexto histórico sobre o desenvolvimento do Bitcoin e defende a utilização do criptoativo como moeda de troca legítimo.

Futuramente, no capítulo \ref*{sec:educativo}, propomos a ideia do desenvolvimento de uma criptomoeda capaz de conteinerizar e abstrair os conceitos básicos do ensino de economia, provendo autonomia para professores simularem ambientes econômicos. Assim demonstrando e aplicando o conhecimento teórico da ciência financeira.



% \section*{METODOLOGIA DE PESQUISA}
\chapter{Metodologia de Pesquisa}
A pesquisa foi conduzida inicialmente através de uma extensa revisão sistemática da literatura, diante das revistas acadêmicas ACM, \textit{Semantic Scholar}, IEEE, em busca de artigos apresentando o estado da arte na área de criptoativos. Foram coletados 86 artigos selecionados inicialmente pelas palavras chaves Criptoativos, \textit{Bitcoin}, \textit{Smart Contracts}, \textit{Blockchain}, \textit{DeFi} e \textit{Web 3}. Utilizamos a revisão destes artigos de modo a buscar o estado da arte documentado diante dos criptoativos.

Após a coleta inicial dos artigos, o primeiro processo de filtragem se passou pela leitura do resumo dos artigos, mantendo apenas os documentos que retratavam a utilização e o impacto socio-econômico dos criptoativos e a exploração do conceito de finanças descentralizadas. Esta primeira filtragem nos retornou 58 registros. 

O segundo processo de filtragem passou-se pela leitura do conteúdo de cada artigo, buscando apenas os relacionamentos dentre os termos técnicos de economia paralelamente ao funcionamento dos ativos digitais. Diante deste processamento de documentação, foi constatada a necessidade da busca bibliográfica de referências da Escola Austríaca de Economia — devido à semelhança de comportamento agnóstico ao estado tanto desta vertente acadêmica quanto das finanças descentralizadas — no que foi considerada busca de literatura de seus principais autores. Esta segunda filtragem nos retornou 7 livros e 27 artigos.

Por fim, a partir da leitura de toda a documentação coletada, foi registrado o estado da arte na utilização de criptoativos, dentre seus serviços, propostas e ferramentas. Foi registrada também a atuação técnica da moeda digital mais utilizada atualmente, o \textit{Bitcoin}\footnote{Conforme o indexador de criptoativos \href{https://www.coingecko.com/pt}{\textit{CoinGecko}}, acessado 22/05/2024}, registrado o comportamento do ativo diante da visão da Escola Austríaca de Economia e documentado as comparações dos ativos digitais diante do ouro e o papel-moeda.











% \section*{ESTADO DA ARTE}
\chapter{Estado da Arte}
O estudo dos criptoativos, frequentemente referidos como criptomoedas, emergiu como uma área interdisciplinar que engloba finanças, economia, ciência da computação, direito e outras disciplinas correlatas. O estado da arte nesta área é caracterizado por avanços tecnológicos inovadores, desafios regulatórios complexos, crescente adoção institucional e individual, e um cenário de rápida evolução e inovações contínuas.

% O estado da arte dos criptoativos é marcado por uma confluência de avanços tecnológicos, desenvolvimentos regulatórios, crescente adoção e contínuas inovações. À medida que a tecnologia blockchain evolui e a regulamentação se torna mais clara, os criptoativos têm o potencial de transformar significativamente o panorama financeiro global, introduzindo novos paradigmas de descentralização, segurança e eficiência. A pesquisa contínua e a colaboração entre setores serão essenciais para navegar os desafios e aproveitar as oportunidades apresentadas por esta tecnologia emergente.

\section{Tecnologia}
A seguir consta as principais tecnologias aplicadas ao ambiente das finanças distribuídas, será apresentado brevemente seus respectivos contextos e utilizações.

\section*{Blockchain e Tecnologia de Registro Distribuído (DLT)}

A \textit{blockchain}, uma forma específica de Tecnologia de Registro Distribuído (DLT), é o alicerce sobre o qual a maioria dos criptoativos é construída. Sua estrutura descentralizada, que permite a verificação e registro de transações por uma rede distribuída de nós, assegura características fundamentais como imutabilidade, transparência e resistência à censura \cite{Nakamoto2009}. Além disso, protocolos de consenso como \textit{Proof of Work (PoW)} e \textit{Proof of Stake (PoS)} são cruciais para a segurança e eficiência das redes \textit{blockchain}.

\section*{Contratos Inteligentes}

Contratos inteligentes \textit{(smart contracts)} são programas autoexecutáveis que operam quando condições predefinidas são atendidas. Introduzidos pela plataforma Ethereum \cite{buterin2013ethereum}, esses contratos têm o potencial de automatizar e desintermediar uma vasta gama de transações e processos contratuais, desde serviços financeiros até cadeias de suprimentos. A segurança e a flexibilidade proporcionadas pelos contratos inteligentes incentivam a criação de aplicações descentralizadas \textit{(DApps)}, que operam em redes \textit{blockchain} sem necessidade de intermediários confiáveis.

\section*{Interoperabilidade}

Um dos desafios técnicos significativos é a interoperabilidade entre diferentes blockchains. Projetos como Polkadot \cite{wood2016polkadot} e Cosmos \cite{kwon2016cosmos} estão na vanguarda do desenvolvimento de soluções que permitem a comunicação e a interação entre diversas redes \textit{blockchain}. Esta interoperabilidade é essencial para a criação de um ecossistema de criptoativos mais integrado e funcional, onde ativos e informações podem ser transferidos de maneira segura e eficiente entre diferentes plataformas.

\section{Regulamentação}

Nesta seção é documentada as respostas de maior incisão no ambiente cripto por parte do âmbito governamental. 

\section*{Panorama Regulatório}

O cenário regulatório dos criptoativos é altamente diversificado e dinâmico. Países como Suíça e Singapura têm adotado abordagens regulatórias favoráveis, criando ambientes propícios para inovação e atração de investimentos \cite{zohar2015bitcoin}. Em contraste, nações como China e Índia têm implementado restrições rigorosas ao uso e comércio de criptoativos, citando preocupações com a estabilidade financeira e a proteção ao consumidor \cite{auer2018regulating}.

\section*{Iniciativas Globais}

Instituições internacionais, como o \textit{Financial Action Task Force (FATF)}, estão desenvolvendo diretrizes para mitigar os riscos associados aos criptoativos, como a lavagem de dinheiro e o financiamento do terrorismo \cite{fatf2019guidance}. A União Europeia, com sua proposta de Regulamento de Mercados de Criptoativos (MiCA), afirma buscar estabelecer um quadro regulamentar abrangente que ofereça proteção aos investidores enquanto promove a inovação no setor \cite{european2020proposal}.

\section{Adoção}

Nesta seção é constatado o âmbito popular e empresarial diante da adoção do uso de criptoativos em seu cotidiano.

\section*{Adoção Institucional}

A adoção de criptoativos por instituições financeiras e empresas de grande porte tem crescido substancialmente. Empresas como \textit{Tesla} e \textit{MicroStrategy} têm incorporado Bitcoin em suas estratégias de reserva de tesouraria, sinalizando uma crescente aceitação dos criptoativos como reserva de valor \cite{bouri2017hedge}. Além disso, grandes instituições financeiras estão desenvolvendo produtos de investimento baseados em criptoativos, como fundos negociados em bolsa (ETFs) e contratos futuros.

\section*{Adoção pelo Consumidor}

A adoção de criptoativos por consumidores está aumentando, impulsionada pela facilidade de acesso através de carteiras digitais e plataformas de negociação \cite{kondor2014do}. Serviços de pagamento como \textit{PayPal} e \textit{Square} permitem a compra, venda e uso de criptomoedas, tornando-as mais acessíveis ao público. Esta crescente adoção está ligada à busca por alternativas ao sistema financeiro tradicional e à percepção de criptoativos como uma forma de investimento ou proteção contra a inflação.

\section{Inovação}
\section*{Finanças Descentralizadas (DeFi)}

O movimento de Finanças Descentralizadas (textit{DeFi}) representa uma das áreas mais inovadoras dentro do ecossistema de criptoativos. Plataformas \textit{DeFi} como \textit{Uniswap}, \textit{Aave} e \textit{Compound} permitem a realização de serviços financeiros como empréstimos, trocas e investimentos de maneira descentralizada, sem a necessidade de intermediários tradicionais \cite{zhang2020data}. Esses serviços são executados através de contratos inteligentes, oferecendo maior transparência e acessibilidade.

\section*{\textit{Tokens} Não Fungíveis (NFTs)}

Os \textit{Tokens} Não Fungíveis (NFTs) emergiram como uma nova classe de ativos digitais que representam a propriedade de itens únicos, como arte digital, música e colecionáveis. A explosão da popularidade dos NFTs em 2021 trouxe atenção significativa para o potencial de textit{tokenização} de ativos e a criação de mercados digitais \cite{wang2021non}. Esta inovação está transformando a maneira como os direitos de propriedade e a escassez digital são percebidos e geridos.

\section*{Web 3.0}

A Web 3.0, também conhecida como internet descentralizada, é uma visão que busca redefinir a estrutura da internet, permitindo maior controle e propriedade de dados pelos usuários. Criptoativos e \textit{DApps (Descentralized Apps)} são componentes centrais desta visão, proporcionando uma infraestrutura para uma web mais segura, transparente e centrada no usuário \cite{zhang2019secure}.



% \section*{DESENVOLVIMENTO}
\chapter{Conceito de possível aplicação da arquitetura de criptoativos}

\section*{Um criptoativo educativo}
\label{sec:educativo}











\section*{CONCLUSÃO}
Neste projeto, percorremos por meio do contexto técnico e social do desenvolvimento dos criptoativos, averiguamos a motivação de desintermediação das atividades financeiras por meio da origem do \textit{Bitcoin}; exploramos o ecossistema cripto por meio dos contratos inteligentes e da \textit{DeFi}, e exibimos seu impacto diante da economia tradicional.

Sob maior incisão teórica no projeto, apresentamos os conceitos chave do funcionamento do \textit{Bitcoin}, constatamos a estrutura dos blocos, a estrutura do encadeamento, o procedimento de prova de trabalho utilizado na mineração do \textit{Bitcoin} e como constitui o seu caráter criptográfico.

Vimos também como a natureza do \textit{Bitcoin}, de maneira programática, corrobora aos conceitos técnico-economicos da Escola Austríaca de Economia, provendo respaldo diante das características do dinheiro, tão bem como as funções da moeda. Apresentamos o resultado da comparação de suas características ao papel-moeda e ao ouros por Fernando Ulrich.

Por fim, atuamos na experimentação educacional da implementação de um sistema de \textit{blockchain} de nível médio, que abstrai os conceitos base do \textit{Bitcoin}, permite a manipulação via requisições e exibe as classes presentes da corrente no navegador. Foi exposto todas as estruturas geradas neste experimento, seus atributos e suas funções diantes da rede.

\section*{Exibição do Experimento de \textit{Blockchain} em Rust}
Nas Figuras a seguir constam as imagens desta experimentação, o repositório onde o código e os arquivos deste teste foram armazenados na plataforma Github por preferência dos autores. O código se encontra disponível por este link:
\href{https://github.com/HeberUnifil/Educational-Blockchain-Currency}{\textcolor{blue}{https://github.com/HeberUnifil/Educational-Blockchain-Currency}} .

Na Figura \ref*{fig:terminal} consta o registro via terminal Linux das atualizações realizadas na \textit{Blockchain}:

\begin{figure} [h]
	\centering
	\caption{Resposta via terminal sobre as atualizações na \textit{Blockchain}.}
	\includegraphics[width=1\linewidth]{../images/terminal-blockchain.png}
	\label{fig:terminal}
	\text{Fonte: Captura de tela tirada pelos autores}

\end{figure}

Na Figura \ref*{fig:navegador} consta a exibição geral dos blocos ordenados na corrente, seus respectivos \textit{hashes}, sua marca temporal, quantidade de transações e o número de tentativas para mineração do bloco.

\begin{figure} [h]
	\centering
	\caption{Resposta via terminal sobre as atualizações na \textit{Blockchain}.}
	\includegraphics[width=1\linewidth]{../images/navegador-blockchain.png}
	\label{fig:navegador}
	\text{Fonte: Captura de tela tirada pelos autores}

\end{figure}

% \begin{figure} [h]
% 	\centering
% 	\caption{Resposta via terminal sobre as atualizações na \textit{Blockchain}.}
% 	\includegraphics[width=.8\linewidth]{../images/terminal-blockchain.png}
% 	\label{fig:terminal}
% 	\text{Fonte: Captura de tela tirada pelos autores}

% \end{figure}

\end{Spacing}
\postextual

% bibliografia %
\bibliography{bibliografia}

\end{document}