% Organização sugerida para uma introdução de projeto de pesquisa

\chapter{Introdução}

Os criptoativos, também conhecidos como criptomoedas ou moedas digitais, são representações digitais de valor que utilizam criptografia para garantir transações seguras e controlar a criação de novas unidades \cite{Yetmar2023}. Ao contrário das moedas governamentais tradicionais, os criptoativos operam em redes descentralizadas, geralmente baseadas em tecnologia de \textit{Blockchain}, onde a validação das transações é realizada pelos próprios participantes da rede através de um consenso distribuído. Este ambiente transparente, seguro e descentralizado existe graças a arquiteturas de \textit{software} complexas e dedicadas ao propósito de escalabilidade, segurança e autonomia do ativo. Diante da versatilidade dos criptoativos, este projeto de pesquisa visa esclarecer o entendimento popular dos criptoativos enquanto expõe uma análise técnica e econômica do mesmo. 

Na seção \ref*{sec:bitcoin} utilizamos do \textit{Bitcoin} como exemplo, expomos o contexto de aplicação deste ativo, seu posicionamento sobre o dinheiro tradicional e como podemos utilizar do ambiente das criptos para buscar independência monetária. É apresentado também a tecnologia de encadeamento e rede em que o Bitcoin atua — a \textit{Blockchain}, a tecnologia Prova de Trabalho, ou \textit{Proof of work}, utilizada como validador de novos blocos e a técnica utilizada na assinatura de transações dentro da \textit{Blockchain} — a criptografia de chave pública e privada.

Pela seção \ref*{sec:austriaca} são adotadas as definições de dinheiro, economia e capital da Escola Austríaca de Economia. Será utilizado dos conceitos financeiros para definir quão bem o criptoativo consegue servir as atividades econômicas.

Durante a seção \ref*{sec:dinheiro} é feita uma revisão do livro ``\textit{Bitcoin}, A Moeda Na Era Digital'' do mestre brasileiro em economia Fernando Ulrich. Neste livro o autor também revisa o posicionamento da Escola Austríaca de Economia diante do conceito de dinheiro e moeda, traz contexto histórico sobre o desenvolvimento do \textit{Bitcoin} e defende a utilização do criptoativo como moeda de troca legítimo.


Por fim, no capítulo \ref*{sec:educativo}, propomos a ideia do desenvolvimento de uma criptomoeda capaz de conteinerizar e abstrair os conceitos básicos do ensino de economia, provendo autonomia para professores simularem ambientes econômicos. Assim demonstrando e aplicando o conhecimento teórico da ciência financeira.

% PROBLEMATICA

Embora as criptomoedas tenham ganhado destaque, a compreensão abrangente de seu estado técnico atual permanece fragmentada. Esta falta de clareza popular diante do contexto de criptoativos torna o conceito menos palatável à aceitação pública da tecnologia. Embora compreendível a baixa adesão popular a tal tecnologia, é possível estipular melhorias de qualidade de vida diante da população passando despercebidas.

Como conceito, as finanças descentralizadas nasceram visando denunciar as consequências sofridas pela população devido ao mal uso governamental do curso forçado das suas respectivas moedas. A ideia de retirar a manipulação central do dinheiro cria impeditivos físicos ao ativo de sofrer anomalias econômicas como a inflação, por exemplo, visto que o comportamento de escassez da moeda é absoluto (no caso do \textit{Bitcoin}).

Conforme o mestre em desenvolvimento econômico Pedro Lopes Marinho, em 2001, devido o início da Primeira Guerra Mundial, o sistema monetário padrão-ouro foi mundialmente abolido enquanto governos financiavam os gastos militares a partir da emissão de moedas. Uma vez que o lastro em metal na moeda foi abandonado, o maior impeditivo a impressão deliberada de dinheiro — e posteriormente inflação — foi deixado de lado.

A ideia da economia estar sob o controle absoluto governamental implica que todo o trabalho, tempo, esforço e riqueza de uma população está a uma ordem de distância de ser descartada por mal uso estatal da sua moeda. 

Assim que as finanças descentralizadas tomam seu devido foco. Uma vez oferecendo independência, autonomia, transparência e integridade dos seus protocolos, as moedas digitais podem garantir que o poder e a responsabilidade do dinheiro estão apenas sob quem os detém, seguindo a máxima popular do ambiente cripto ``Minhas chaves, minhas moedas''.

% Limitações do trabalho

O projeto de pesquisa passa pela limitação de que, no estado atual de atividade entre meio destes ativos, o lançamento de novas moedas digitais são extremamente frequentes, gerando constante necessidade de reindexação da funcionalidade e atuação de cada nova moeda.
