\chapter{Fundamentação Teórica}
Ao decorrer desta pesquisa utilizamos como modelo de criptoativo o \textit{Bitcoin}, criado pelo pseudônimo Satoshi Nakamoto. Esta moeda é, atualmente, a mais estável diante do mercado de criptoativos e a mais antiga também, percorrendo desde 2008. É manifesto neste trabalho o contexto em que o \textit{Bitcoin} foi criado, seus objetivos diante da população e o detalhamento das tecnologias em que o ativo foi forjado.

Diante da tecnologia empregada no \textit{Bitcoin}, neste projeto é evidenciado os pilares principais em que se apoiam o desenvolvimento das criptomoedas, definido o trilema das criptos e como este trilema impacta na execução das suas determinadas funções.

Consta também apresentado neste capítulo o conceito de \textit{DeFi} — Sigla para Finanças Descentralizadas em inglês — e como este ecossistema tecnológico pode prover maior qualidade de vida e serviços para seus usuários.

Este projeto utiliza dos autores da Escola Austríaca de Economia — Ludwig von Mises, Böm Bawerk, Carl Menger, Friedrich Hayek e Murray Rothbard — para induzir a definição de dinheiro e moeda, pois, com foco na definição do conceito de dinheiro é possível argumentar o quão bem uma criptomoeda cumpre este papel em comparação com a moeda de curso legal do estado.

\section{O Bitcoin} \label{sec:bitcoin}
O \textit{Bitcoin} nasceu como um modelo de dinheiro digital que opera em uma rede descentralizada, sem a necessidade de uma autoridade central para emitir ou controlar a moeda. Foi proposto pela primeira vez em 2008 pelo programador anônimo conhecido pelo pseudônimo Satoshi Nakamoto, na documentação \textit{"Bitcoin: A Peer-to-Peer Electronic Cash System"} \cite{Nakamoto2009} ,e lançado como \textit{software} de código aberto em 2009. O \textit{Bitcoin} permite transações \textit{peer-to-peer} — de pessoa para pessoa e/ou ponto a ponto —, nas quais os usuários podem enviar e receber pagamentos diretamente, sem a necessidade de intermediários.

Este criptoativo é reconhecido por ser o primeiro e mais estável projeto de moeda digital e é definitivamente visto como referência de segurança, escalabilidade e descentralização no ambiente cripto.

Posteriormente há uma melhor definição da tecnologia de registro em que o Bitcoin atua, a \textit{Blockchain}.

\section*{A Blockchain} \label{subsec:blockchain}
A tecnologia fundamental que sustenta o \textit{Bitcoin} é a \textit{Blockchain}, um livro-razão digital público e distribuído que registra todas as transações de forma transparente e imutável. A \textit{Blockchain} é composta por blocos encadeados de forma cronológica. De maneria recursiva, cada bloco contém um conjunto — por ordem temporal — de transações confirmadas de maneira encadeada e um cabeçalho que inclui um \textit{hash} do bloco anterior, formando assim uma cadeia de blocos interligados.

Na Figura \ref*{fig:blockchain} exibimos uma abstração do encadeamento de blocos e seus dados diante da \textit{Blockchain} formando a estrutura e formato da cadeia.

\begin{figure} [h]
	\centering
	\caption{Estrutura de encadeamento de blocos numa \textit{blockchain}.}
	\includegraphics[width=.8\linewidth]{images/figura 2.png}
	\label{fig:blockchain}
	\text{Fonte: Tradução pelos autores, baseado na documentação de referência \cite{Nakamoto2009}}
\end{figure}

Na Figura \ref*{fig:transactions} exibimos uma abstração da perspectiva do bloco, onde ocorre o encadeamento das transações seguido das suas respectivas assinaturas.

\begin{figure} [h]
	\centering
	\caption{Encadeamento das transações nos blocos.}
	\includegraphics[width=.8\linewidth]{images/figura 1.png}
	\label{fig:transactions}
	\text{Fonte: Tradução pelos autores, baseado na documentação de referência \cite{Nakamoto2009}}

\end{figure}

Em seguida é explicada a tecnologia de criptografia utilizada na atuação do \textit{Bitcoin}.

\section*{Criptografia SHA-256 (Secure Hash Algorithm 256-bit)} \label{subsec:sha256}
O protocolo do Bitcoin emprega a criptografia SHA-256 (\textit{Secure Hash Algorithm 256-bit}) como um componente fundamental para garantir a integridade e a segurança das transações na rede. Este algoritmo de \textit{hash} criptográfico, desenvolvido pela Agência Nacional de Segurança (NSA) dos Estados Unidos e publicado pelo Instituto Nacional de Padrões e Tecnologia (NIST), é crucial para diversas operações no ecossistema Bitcoin.

O SHA-256 contribui para a segurança geral do protocolo Bitcoin por ser resistente a ataques de colisão e de pré-imagem, o que significa ser computacionalmente impraticável encontrar duas mensagens distintas que resultem no mesmo \textit{hash} ou reverter um \textit{hash} para obter a mensagem original. Essas propriedades são essenciais para manter a integridade das chaves e transações, garantindo que as entradas não possam ser manipuladas sem que seja facilmente detectado pela rede.

\section*{Algoritmo de Prova de Trabalho e Mineiração} \label{subsec:pow}
Para garantir a segurança e a integridade da \textit{Blockchain}, o \textit{Bitcoin} utiliza um algoritmo de consenso chamado Prova de Trabalho (tambem chamado de \textit{Proof of Work} ou PoW). Os mineradores coletam transações pendentes em um bloco e tentam gerar um \textit{hash} válido para esse bloco usando o SHA-256.

A prova de trabalho se orienta dentro da \textit{blockchain} diante do protocolo Merkle. O protocolo Merkle, também referido como árvore de Merkle ou \textit{hash} de Merkle, é uma estrutura de dados fundamental em criptografia, criada por Ralph Merkle. A árvore de Merkle ajuda a garantir que os dados não foram alterados, pois, qualquer modificação nos dados de entrada alteraria o \textit{hash} na folha correspondente e, por sua vez, todos os \textit{hashes} no caminho até a raiz.


O algoritmo é aplicado duas vezes (conhecido como \textit{double-SHA-256}) ao cabeçalho do bloco, que inclui a versão do programa, o \textit{hash} do bloco anterior, o \textit{hash} Merkle das transações no bloco, o \textit{timestamp}, o nível de dificuldade e um \textit{nonce}. O termo \textit{nonce} refere-se a um número que é usado apenas uma vez (do inglês \textit{number used once}). O \textit{nonce} é um valor inteiro de 32 bits que os mineradores ajustam repetidamente para tentar produzir um \textit{hash} do bloco que atenda aos critérios de dificuldade estabelecidos pela rede Bitcoin.

O objetivo é encontrar um \textit{hash} que seja menor que o valor de dificuldade estabelecido pela rede, o que exige que os mineradores ajustem o \textit{nonce} repetidamente e recalculam o \textit{hash} do bloco até que um valor adequado seja encontrado. Este processo é fundamental para a implementação da prova de trabalho (\textit{Proof of Work} - PoW), que ajuda a proteger a rede contra ataques.

\clearpage
\section{Escola Austríaca de Economia} \label{sec:austriaca}
A Escola Austríaca de Economia, emergida no final do século XIX, representa uma tradição heterodoxa significativa no pensamento econômico. Caracterizada por sua ênfase na teoria subjetiva do valor, na praxeologia e no papel crucial do empreendedorismo, a Escola Austríaca oferece uma perspectiva única que contrasta com as abordagens neoclássicas e keynesianas dominantes.

Seguidamente é revisitada a história, cronologia, os principais autores e pautas centrais desta escola, destacando suas contribuições teóricas e influências duradouras. É constatado também definições por parte destes autores diante dos conceitos, respectivamente, de economia, capital e dinheiro

% \section*{Origens}

A Escola Austríaca de Economia foi fundada por Carl Menger com a publicação de \textit{"Principles of Economics"} (1871). Seu trabalho desafiou a teoria do valor-trabalho dos economistas clássicos e introduziu a teoria marginalista do valor.

\section*{Carl Menger (1840-1921)}
Em \textit{"Principles of Economics"}, \cite{menger1871principles} argumentou que o valor dos bens é determinado pela utilidade marginal que os indivíduos o atribuem, estabelecendo as bases para a análise econômica subjetiva.

\section*{Eugen von Böhm-Bawerk (1851-1914)}
Discípulo de Menger, Böhm-Bawerk contribuiu significativamente para a teoria do capital e dos juros. Em \textit{"Capital and Interest"}, ele desenvolveu a teoria da estrutura temporal da produção, enfatizando a importância do tempo no processo produtivo \cite{bohm1884capital}.

\section*{Friedrich von Wieser (1851-1926)}
Wieser é conhecido por sua teoria do custo de oportunidade e pelo desenvolvimento da teoria do valor imputado. Ele ajudou a consolidar a Escola Austríaca como uma corrente de pensamento econômico significativa \cite{bohm1884capital}.

\section*{Desenvolvimento e Consolidação}

No início do século XX, Ludwig von Mises e Friedrich Hayek ampliaram e consolidaram as ideias da Escola Austríaca, influenciando significativamente o pensamento econômico.

\section*{Ludwig von Mises (1881-1973)}
Mises é uma figura central na Escola Austríaca. Em \textit{"Human Action"}, ele propôs que a economia deve ser baseada na lógica dedutiva da ação humana, uma abordagem chamada praxeologia. Mises também desenvolveu a teoria do ciclo econômico, que analisa as flutuações econômicas causadas pela expansão do crédito e pela intervenção estatal no mercado monetário \cite{mises1949human}.

\section*{Friedrich Hayek (1899-1992)}
Discípulo de Mises, Hayek contribuiu para a teoria do capital e o estudo dos ciclos econômicos. Em \textit{"The Road to Serfdom"} e \textit{"The Constitution of Liberty"}, Hayek criticou o intervencionismo estatal e defendeu uma ordem espontânea de mercado. Em 1974, ele recebeu o Prêmio Nobel de Economia por seu trabalho sobre a teoria monetária e as flutuações econômicas \cite{hayek1944road},\cite{hayek1960constitution}.

\section*{Expansão e Influência Contemporânea}

Na segunda metade do século XX e início do século XXI, a Escola Austríaca continuou a evoluir com novos pensadores que expandiram suas teorias e influências.

\section*{Murray Rothbard (1926-1995)}
Rothbard combinou a economia austríaca com uma filosofia libertária. Em \textit{"Man, Economy, and State"}, ele apresentou uma visão abrangente da economia austríaca e criticou a intervenção estatal. \cite{rothbard1962man}.

\section*{Israel Kirzner (1930-)}
Kirzner contribuiu para a teoria do empreendedorismo, destacando o papel do empreendedor na descoberta de oportunidades de mercado e na coordenação econômica. Seu trabalho \textit{"Competition and Entrepreneurship"} é uma referência importante para o estudo do processo de mercado e da função empresarial \cite{kirzner1973competition}.

\section*{Pautas e Contribuições}
A seguir consta as principais pautas da Escola Austríaca aonde servirá de base para o relacionamento dos conceitos fundamentais de economia, dinheiro e capital.

\section*{Teoria do Valor Subjetivo}

A teoria do valor subjetivo é uma contribuição fundamental da Escola Austríaca. Ela afirma que o valor dos bens é determinado pela utilidade marginal atribuída pelos indivíduos, contrastando com a teoria do valor-trabalho \cite{menger1871principles}.

\section*{Praxeologia}

A praxeologia é a metodologia central da Escola Austríaca. Baseia-se na premissa de que a economia é uma ciência social que deve ser estudada através da análise lógica da ação humana, ao invés de métodos empíricos e estatísticos \cite{mises1949human}.

\section*{Teoria do Ciclo Econômico}

A teoria austríaca do ciclo econômico, desenvolvida por Mises e Hayek, explica as flutuações econômicas como resultado das distorções causadas pela expansão do crédito e pela intervenção governamental. Segundo esta teoria, a criação artificial de crédito leva a um mau investimento de recursos, resultando em ciclos de \textit{boom} e \textit{bust} \cite{mises1949human,hayek1944road}.

\section*{Crítica ao Intervencionismo Estatal}

A Escola Austríaca é fortemente crítica ao intervencionismo estatal e ao planejamento centralizado. Economistas austríacos argumentam que a intervenção governamental distorce os sinais de preço, leva a alocações ineficientes de recursos e restringe a liberdade individual. Hayek argumentou que o planejamento centralizado é incapaz de lidar com a complexidade da informação distribuída na sociedade \cite{hayek1944road,hayek1960constitution}.

\section*{Teoria do Empreendedorismo}

A ênfase no papel do empreendedor é uma característica distintiva da Escola Austríaca. Kirzner destacou que os empreendedores são essenciais para a descoberta e exploração de oportunidades de mercado, contribuindo para a coordenação econômica e a dinâmica dos mercados \cite{kirzner1973competition}.

\section{Definição conceitual de maior incisão no projeto}

Consta aqui definições por parte dos autores austríacos diante dos conceitos de base da atuação deste artigo. A partir desta conceitualização é possível observar diante da atuação dos criptoativos e futuramente estipular as necessidades de um sistema de moedas digitais, dedicado ao ensino de finanças.

\section*{Economia e a Ação Humana}
Ludwig von Mises, em "Ação Humana"\cite{mises1949human}, define economia como "a ciência que estuda a ação humana, uma aplicação da teoria do conhecimento humano". Segundo Mises, a economia é um ramo da praxeologia, ou seja, a teoria da ação humana. Ele argumenta que a economia, ao contrário de ser meramente uma análise de dados e tendências, é fundamentalmente sobre como os indivíduos escolhem agir com recursos escassos para atingir seus objetivos.

\section*{Capital segundo Böhm-Bawerk}
Eugen Böhm von Bawerk, contribuiu significativamente para a teoria do capital. Em sua obra \textit{"Capital and Interest"}\cite{bohm1884capital}, Böhm-Bawerk descreve o capital como "bens produzidos que servem como meios para a aquisição de bens futuros" \cite{bohm1884capital}. Ele esclarece que o capital não é simplesmente uma acumulação de dinheiro ou ativos, mas sim ferramentas, máquinas e materiais usados para aumentar a produção futura.

\section*{Dinheiro e sua Origem para Menger}
Carl Menger, foi um dos primeiros economistas a explicar a origem do dinheiro através de um processo de evolução social e não por decreto governamental ou convenção. Em sua obra "Princípios de Economia Política" \cite{menger2017liberalismo}, Menger argumentou que o dinheiro emergiu organicamente como o meio mais vendável de troca, facilitando assim as transações comerciais e reduzindo os custos de transação na economia \cite{menger1871principles}.

\section*{Hayek e a Desestatização do Dinheiro}
Friedrich Hayek, levou a teoria monetária austríaca para outra direção ao argumentar a favor da competição de moedas privadas em sua obra "Desnacionalização do Dinheiro" \cite{hayek2017desestatizaccao}. Hayek criticou os monopólios governamentais sobre a emissão de dinheiro, propondo que a concorrência entre diferentes categorias de dinheiro poderia prevenir a inflação e promover a estabilidade econômica.

\section*{A Teoria do Dinheiro de Mises}
Ludwig von Mises expandiu a teoria de Menger ao introduzir o conceito de "regressão" em sua análise do valor do dinheiro. Em "A Teoria do Dinheiro e do Crédito" \cite{von2013theory}, Mises apresenta a ideia de que o valor do dinheiro hoje é derivado da expectativa de seu poder de compra no futuro, que por sua vez é baseado em uma regressão contínua até o ponto em que o dinheiro era apenas um bem mais vendável entre outros \cite{von2013theory}. Mises também destacou o papel do dinheiro no cálculo econômico, essencial para a alocação racional de recursos em uma economia de mercado.

\section*{Rothbard e a Crítica à Moeda Fiduciária}
Murray Rothbard, seguindo a indução de Mises, foi crítico em relação ao sistema de moeda fiduciária e ao papel dos bancos centrais. Em "O que o Governo fez com o Nosso Dinheiro?" \cite{rothbard2022governo}, Rothbard explica como o dinheiro historicamente ancorado em commodities, como o ouro, foi progressivamente substituído por dinheiro papel sem lastro, levando a ciclos econômicos mais instáveis e inflação.

\section{Bitcoin é dinheiro de verdade?} \label{sec:dinheiro}
Nesta secção, fazemos uma revisão do livro "\textit{Bitcoin}, a moeda na era digital"\cite{Ulrich2014} escrito por Fernando Ulrich. O brasileiro é Mestre em Economia e referência por seu pioneirismo na divulgação de criptomoedas no Brasil.

\section*{Definição de Ulrich de Dinheiro e Moeda}
Em seu livro, Ulrich chega a definição de moeda como "qualquer bem econômico empregado indefinidamente como meio de troca, independentemente de sua liquidez frente a outros bens monetários e de seus possíveis usos alternativos" \cite[P.89]{Ulrich2014}.

O autor lista atributos característicos a moeda, sendo eles sua escassez,
durabilidade, homogeneidade espacial e temporal, divisibilidade e maleabilidade, comparando o desempenho destes atributos diante do papel-moeda, o ouro e o Bitcoin, como mostra na Tabela \ref*{tab:atributos}.

% \begin{figure} [h]
% 	\centering
% 	\caption{Comparação dos atributos do dinheiro diante do ouro, do papel-moeda e do \textit{Bitcoin}}
% 	\includegraphics[width=.8\linewidth]{images/atributos.png}
% 	\label{fig:atributos}
% 	\text{Fonte: \cite[p.67]{Ulrich2014}.}
% \end{figure}
% Please add the following required packages to your document preamble:
% \usepackage[table,xcdraw]{xcolor}
% Beamer presentation requires \usepackage{colortbl} instead of \usepackage[table,xcdraw]{xcolor}
\FloatBarrier
\begin{table}[h]
    \centering
	\caption{Comparação dos atributos do dinheiro diante do ouro, do papel-moeda e do \textit{Bitcoin}}.
	\begin{tabular}{|c|c|c|c|}
        
		\hline
		\rowcolor[HTML]{C0C0C0}
		\textbf{Atributos}                                                                & \textbf{Ouro}                                                     & \textbf{Papel-moeda}                                                              & \textbf{Bitcoin}                                                    \\ \hline
		1.Durabilidade                                                                    & Alta                                                              & Baixa                                                                             & Perfeita                                                            \\ \hline
		2.Divisibilidade                                                                  & Média                                                             & Alta                                                                              & Perfeita                                                            \\ \hline
		3.Maleabilidade                                                                   & Alta                                                              & Alta                                                                              & Incorpóreo                                                          \\ \hline
		4.Homogeneidade                                                                   & Média                                                             & Alta                                                                              & Perfeita                                                            \\ \hline
		5.Oferta(Escassez)                                                                & \begin{tabular}[c]{@{}c@{}}Limitada pela \\ natureza\end{tabular} & \begin{tabular}[c]{@{}c@{}}Limitada e \\ controlada \\ politicamente\end{tabular} & \begin{tabular}[c]{@{}c@{}}Limitada \\ Matematicamente\end{tabular} \\ \hline
		\begin{tabular}[c]{@{}c@{}}6.Dependência de \\ terceiros fiduciários\end{tabular} & Alta                                                              & Alta                                                                              & \begin{tabular}[c]{@{}c@{}}Baixa ou \\ quase nula\end{tabular}      \\ \hline
	\end{tabular}
    \label{tab:atributos}
\end{table}
\FloatBarrier

Ulrich menciona também as funções do dinheiro, listadas de servir como meio de troca, reserva de valor e unidade de conta. Em outras palavras, uma moeda deve servir, respectivamente, de maneira que as suas trocas sejam de forma facilitada; deve atuar de maneira em que possa ser entesourada e/ou guardada como reserva de riqueza; e por fim permita ser utilizável como meio de conta, utilizável ao cálculo econômico em função da moeda.

Segundo Fernando, o \textit{Bitcoin} mostra-se capaz de performar as características e as funções da moeda tão bem, se não melhor, que o ouro e o papel-moeda. De acordo com ele "apesar da aparência unicamente digital, as atuais formas de dinheiro assemelham-se em muito ao \textit{Bitcoin}. A maior parte da massa monetária no mundo moderno manifesta-se de forma intangível; nosso dinheiro já é um bem incorpóreo, uma característica que em nada nos impede de usá-lo diariamente"\cite[p.95]{Ulrich2014}.


% Descrição do estado da arte do tema. O estudante deve demonstrar conhecimento e embasamento. Entre 5 a 10 referências.
