\chapter{Problemática da pesquisa}

% Descrição do problema de pesquisa a ser abordado, hipótese e pré-evidências (tanto do problema quanto para a hipótese).

% Metodologia a ser utilizada para verificar a hipótese, com justificativa.

Embora as criptomoedas tenham ganhado destaque, a compreensão abrangente de seu estado técnico atual permanece fragmentada. Esta falta de clareza popular diante do contexto de criptoativos torna o conceito menos palatável à aceitação pública da tecnologia. Embora compreendível a baixa adesão popular a tal tecnologia, é possível estipular melhorias de qualidade de vida diante da população passando despercebidas.

Como conceito, as finanças descentralizadas nasceram visando denunciar as consequências sofridas pela população devido ao mal uso governamental do curso forçado das suas respectivas moedas. A ideia de retirar a manipulação central do dinheiro cria impeditivos físicos ao ativo de sofrer anomalias econômicas como a inflação, por exemplo, visto que o comportamento de escassez da moeda é absoluto (no caso do \textit{Bitcoin}).

Conforme o mestre em desenvolvimento econômico Pedro Lopes Marinho, em 2001, devido o início da Primeira Guerra Mundial, o sistema monetário padrão-ouro foi mundialmente abolido enquanto governos financiavam os gastos militares a partir da emissão de moedas. Uma vez que o lastro em metal na moeda foi abandonado, o maior impeditivo a impressão deliberada de dinheiro — e posteriormente inflação — foi deixado de lado.

A ideia da economia estar sob o controle absoluto governamental implica que todo o trabalho, tempo, esforço e riqueza de uma população está a uma ordem de distância de ser descartada por mal uso estatal da sua moeda. 

Assim que as finanças descentralizadas tomam seu devido foco. Uma vez oferecendo independência, autonomia, transparência e integridade dos seus protocolos, as moedas digitais podem garantir que o poder e a responsabilidade do dinheiro estão apenas sob quem os detém, seguindo a máxima popular do ambiente cripto ``Minhas chaves, minhas moedas''.
