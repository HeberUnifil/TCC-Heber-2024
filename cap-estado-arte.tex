\chapter{Estado da Arte}
O estudo dos criptoativos, frequentemente referidos como criptomoedas, emergiu como uma área interdisciplinar que engloba finanças, economia, ciência da computação, direito e outras disciplinas correlatas. O estado da arte nesta área é caracterizado por avanços tecnológicos inovadores, desafios regulatórios complexos, crescente adoção institucional e individual, e um cenário de rápida evolução e inovações contínuas.

% O estado da arte dos criptoativos é marcado por uma confluência de avanços tecnológicos, desenvolvimentos regulatórios, crescente adoção e contínuas inovações. À medida que a tecnologia blockchain evolui e a regulamentação se torna mais clara, os criptoativos têm o potencial de transformar significativamente o panorama financeiro global, introduzindo novos paradigmas de descentralização, segurança e eficiência. A pesquisa contínua e a colaboração entre setores serão essenciais para navegar os desafios e aproveitar as oportunidades apresentadas por esta tecnologia emergente.

\section{Tecnologia}
A seguir consta as principais tecnologias aplicadas ao ambiente das finanças distribuídas, será apresentado brevemente seus respectivos contextos e utilizações.

\section*{Blockchain e Tecnologia de Registro Distribuído (DLT)}

A \textit{blockchain}, uma forma específica de Tecnologia de Registro Distribuído (DLT), é o alicerce sobre o qual a maioria dos criptoativos é construída. Sua estrutura descentralizada, que permite a verificação e registro de transações por uma rede distribuída de nós, assegura características fundamentais como imutabilidade, transparência e resistência à censura \cite{Nakamoto2009}. Além disso, protocolos de consenso como \textit{Proof of Work (PoW)} e \textit{Proof of Stake (PoS)} são cruciais para a segurança e eficiência das redes \textit{blockchain}.

\section*{Contratos Inteligentes}

Contratos inteligentes \textit{(smart contracts)} são programas autoexecutáveis que operam quando condições predefinidas são atendidas. Introduzidos pela plataforma Ethereum \cite{buterin2013ethereum}, esses contratos têm o potencial de automatizar e desintermediar uma vasta gama de transações e processos contratuais, desde serviços financeiros até cadeias de suprimentos. A segurança e a flexibilidade proporcionadas pelos contratos inteligentes incentivam a criação de aplicações descentralizadas \textit{(DApps)}, que operam em redes \textit{blockchain} sem necessidade de intermediários confiáveis.

\section*{Interoperabilidade}

Um dos desafios técnicos significativos é a interoperabilidade entre diferentes blockchains. Projetos como Polkadot \cite{wood2016polkadot} e Cosmos \cite{kwon2016cosmos} estão na vanguarda do desenvolvimento de soluções que permitem a comunicação e a interação entre diversas redes \textit{blockchain}. Esta interoperabilidade é essencial para a criação de um ecossistema de criptoativos mais integrado e funcional, onde ativos e informações podem ser transferidos de maneira segura e eficiente entre diferentes plataformas.

\section{Regulamentação}

Nesta seção é documentada as respostas de maior incisão no ambiente cripto por parte do âmbito governamental. 

\section*{Panorama Regulatório}

O cenário regulatório dos criptoativos é altamente diversificado e dinâmico. Países como Suíça e Singapura têm adotado abordagens regulatórias favoráveis, criando ambientes propícios para inovação e atração de investimentos \cite{zohar2015bitcoin}. Em contraste, nações como China e Índia têm implementado restrições rigorosas ao uso e comércio de criptoativos, citando preocupações com a estabilidade financeira e a proteção ao consumidor \cite{auer2018regulating}.

\section*{Iniciativas Globais}

Instituições internacionais, como o \textit{Financial Action Task Force (FATF)}, estão desenvolvendo diretrizes para mitigar os riscos associados aos criptoativos, como a lavagem de dinheiro e o financiamento do terrorismo \cite{fatf2019guidance}. A União Europeia, com sua proposta de Regulamento de Mercados de Criptoativos (MiCA), afirma buscar estabelecer um quadro regulamentar abrangente que ofereça proteção aos investidores enquanto promove a inovação no setor \cite{european2020proposal}.

\section{Adoção}

Nesta seção é constatado o âmbito popular e empresarial diante da adoção do uso de criptoativos em seu cotidiano.

\section*{Adoção Institucional}

A adoção de criptoativos por instituições financeiras e empresas de grande porte tem crescido substancialmente. Empresas como \textit{Tesla} e \textit{MicroStrategy} têm incorporado Bitcoin em suas estratégias de reserva de tesouraria, sinalizando uma crescente aceitação dos criptoativos como reserva de valor \cite{bouri2017hedge}. Além disso, grandes instituições financeiras estão desenvolvendo produtos de investimento baseados em criptoativos, como fundos negociados em bolsa (ETFs) e contratos futuros.

\section*{Adoção pelo Consumidor}

A adoção de criptoativos por consumidores está aumentando, impulsionada pela facilidade de acesso através de carteiras digitais e plataformas de negociação \cite{kondor2014do}. Serviços de pagamento como \textit{PayPal} e \textit{Square} permitem a compra, venda e uso de criptomoedas, tornando-as mais acessíveis ao público. Esta crescente adoção está ligada à busca por alternativas ao sistema financeiro tradicional e à percepção de criptoativos como uma forma de investimento ou proteção contra a inflação.

\section{Inovação}
\section*{Finanças Descentralizadas (DeFi)}

O movimento de Finanças Descentralizadas (textit{DeFi}) representa uma das áreas mais inovadoras dentro do ecossistema de criptoativos. Plataformas \textit{DeFi} como \textit{Uniswap}, \textit{Aave} e \textit{Compound} permitem a realização de serviços financeiros como empréstimos, trocas e investimentos de maneira descentralizada, sem a necessidade de intermediários tradicionais \cite{zhang2020data}. Esses serviços são executados através de contratos inteligentes, oferecendo maior transparência e acessibilidade.

\section*{\textit{Tokens} Não Fungíveis (NFTs)}

Os \textit{Tokens} Não Fungíveis (NFTs) emergiram como uma nova classe de ativos digitais que representam a propriedade de itens únicos, como arte digital, música e colecionáveis. A explosão da popularidade dos NFTs em 2021 trouxe atenção significativa para o potencial de textit{tokenização} de ativos e a criação de mercados digitais \cite{wang2021non}. Esta inovação está transformando a maneira como os direitos de propriedade e a escassez digital são percebidos e geridos.

\section*{Web 3.0}

A Web 3.0, também conhecida como internet descentralizada, é uma visão que busca redefinir a estrutura da internet, permitindo maior controle e propriedade de dados pelos usuários. Criptoativos e \textit{DApps (Descentralized Apps)} são componentes centrais desta visão, proporcionando uma infraestrutura para uma web mais segura, transparente e centrada no usuário \cite{zhang2019secure}.

